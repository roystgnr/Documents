\section{Weighted Residuals}
%% Auto-generate the TOC slide(s)
\begin{frame}
  \tableofcontents[currentsection]
  %\tableofcontents
\end{frame}


\begin{frame}[<+->]
      %\frametitle{Weighted Residual Statement}
  \begin{itemize}
  \item {The point of departure in any FE analysis which uses \libMesh{} is
    the weighted residual statement
    %(sometimes referred to as simply ``the residual'' in
    %the documentation.)
    \begin{equation}
      \nonumber
      (F( u ), v) = 0 \hspace{.5in} \forall v \in \mathcal{V}
    \end{equation}
    }

  \item{ Or, more precisely, the weighted residual statement associated with the
    finite-dimensional space $\mathcal{V}^h \subset \mathcal{V}$
    \begin{equation}
      \nonumber
      (F( u^{\alert{h}} ), v^{\alert{h}}) = 0 \hspace{.5in} \forall v^{\alert{h}} \in \mathcal{V}^{\alert{h}}
  \end{equation}}

  \item{ Even stabilized formulations boil down to some semilinear
          form
    \begin{equation}
      \nonumber
      \Res( u^{\alert{h}}, v^{\alert{h}}) = 0 \hspace{.5in} \forall v^{\alert{h}} \in \mathcal{V}^{\alert{h}}
  \end{equation}}

  \end{itemize}
\end{frame}


\subsection*{Some Examples}    
\begin{frame}[t]
  %\frametitle{Some Examples}
    \begin{block}{
	\only<1-2>{Poisson Equation}
	\only<3-4>{Linear Convection-Diffusion}
	\only<5-6>{Stokes Flow}
      }

      \only<1-2>
      {
	\begin{equation}
	      \nonumber
	      -\Delta u  = f
	      \hspace{.25in} \in \hspace{.1in} \Omega  
	    \end{equation}
      }
      
      \only<3-4>
	  {
	    \begin{equation}
	      \nonumber
	      %\frac{\partial u}{\partial t}
	      -k\Delta u + \bv{b} \cdot \nabla u = f
	      \hspace{.25in} \in \hspace{.1in} \Omega  
	    \end{equation}
	  }

      \only<5-6>
      {
	\begin{equation}
	    \begin{array}{rcl}
	      \nonumber
	      %\frac{\partial \bv{u}}{\partial t} +
	      %\left(\bv{u} \cdot \nabla\right) \bv{u} +
	      \nabla p - \nu \Delta \bv{u}  &=& \bv{f}
	        \\
	      \nonumber
	      \nabla \cdot \bv{u} &=& 0
	    \end{array}  \hspace{.25in}  \in \hspace{.1in} \Omega
	\end{equation}
      }

      
\end{block}
    %\pause

    \only<2,4,6>
    {
    \begin{block}{Weighted Residual Statement}
    }
      \only<2>
      {
      \begin{eqnarray}
	\nonumber
	(F( u ), v) := %\hspace{3in} \\  \nonumber
	\int_{\Omega}  \left[ \nabla u \cdot \nabla v - fv \right] dx \\ \nonumber
	+ \int_{\partial \Omega_N} \left(\nabla u \cdot \bv{n}\right) v \;ds
      \end{eqnarray}
%%       $^{\ast}$ We have employed the divergence theorem to obtain the weighted residual statement.
%%       In general this procedure gives rise to boundary terms which for simplicity we do not discuss
%%       in detail.
      }
      
    \only<4>
    {
      \begin{eqnarray}
	\nonumber
	(F( u ), v) := 
	\int_{\Omega} \left[
	  %\tfrac{\partial u}{\partial t}v  +
	  k\nabla u \cdot \nabla v + (\bv{b} \cdot \nabla u) v - fv \right] dx \\ \nonumber
	+ \int_{\partial \Omega_N} k\left(\nabla u \cdot \bv{n}\right) v \;ds
      \end{eqnarray}
    }

    \only<6>
    {
      \vspace{-.2in}
      \begin{eqnarray}
	\nonumber
	u := \left[\bv{u}, p\right]
	\hspace{.1in},\hspace{.1in}
	v := \left[\bv{v}, q\right]
      \end{eqnarray}
      \vspace{-.25in}
	\begin{eqnarray}
	  \nonumber
	(F( u ), v) := %\hspace{3in} \\ \nonumber
	\int_{\Omega} \left[
	  %\left( \tfrac{\partial \bv{u}}{\partial t}	  +
	  %\left( \bv{u} \cdot \nabla  \right)\bv{u}
	  %\right)
	  %\cdot \bv{v}
	- p\left(\nabla \cdot \bv{v}\right) 
	+ \nu \nabla \bv{u} \colon\!\! \nabla \bv{v} - \bv{f}\cdot \bv{v} \; \right. \\ \nonumber
	+ \left.\left( \nabla \cdot \bv{u} \right) q \right] dx
	+ \int_{\partial \Omega_N} \left(\nu \nabla \bv{u} -p\bv{I}\right)  \bv{n} \cdot \bv{v} \;ds %\hspace{1in}	
      \end{eqnarray}
    }
\only<2,4,6>
{
    \end{block}
 }     
\end{frame} 



%\subsection*{Approximate Problem}
\begin{frame}%[<+->]
  %\frametitle{Weighted Residual Statement}
  \begin{itemize}

    %%   \item{In each of the examples, the weighted residual statement is obtained by
    %%     multiplying the PDE by a test function $v$, integrating over the domain $\Omega$,
    %%     and applying the divergence theorem.}

    %%   \item{Since $v=0$ on $\partial \Omega_D$ (essential data) the boundary integrals
    %%     are over $\partial \Omega_N$ only.}

    %%   \item{There are simple and efficient techniques (e.g.\ penalty method) for
    %%     enforcing the Dirichlet conditions.}

  \item{To obtain the approximate problem, we simply
    replace $u \leftarrow u^h$, $v \leftarrow v^h$, and $\Omega \leftarrow \Omega^h$
    in the weighted residual
    statement.}
    
  \end{itemize}
\end{frame}
