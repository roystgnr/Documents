


\commentout{
%%%%%%%%%%%%%%%%%%%%%%%%%%%%%%%%%%%%%%%%%%%%%%%%%
\frame
{
  \frametitle{Software Reusability}
  \begin{itemize}
    \item At the inception of \libMesh{} in 2002, there were many high-quality software libraries that implemented some aspect of the end-to-end PDE simulation process:
      \begin{itemize}
        \item Parallel linear algebra
        \item Partitioning algorithms for domain decomposition
        \item Visualization formats
        \item \ldots
      \end{itemize}
    \item A design goal of \libMesh{} has always been to provide flexible \& extensible interfaces to existing software whenever possible.
    \item We implement the ``glue'' to these pieces, as well as what we viewed as the missing infrastructure:
      \begin{itemize}
        \item \emphcolor{Flexible data structures for the discretization of spatial domains and systems of PDEs posed on these domains.}
      \end{itemize}          
  \end{itemize}  
}
} % commentout


\section{Library Design}


\begin{frame}
\frametitle{Geometric Element Classes}

\begin{columns}
\column{.55\textwidth}
\begin{center}
\vspace{-5mm}
\includegraphics[width=.75\textwidth]{DofObjects}
\end{center}
\column{.45\textwidth}
\begin{itemize}
\item Abstract interface gives mesh topology
\item Concrete instantiations of mesh geometry
\item Hides element type from most applications
\item Runtime polymorphism allows mixed element types, dimensions
\item Base class data arrays allow more optimization, inlining
\end{itemize}

\end{columns}

\end{frame}


%%%%%%%%%%%%%%%%%%%%%%%%%%%%%%%%%%%%%%%%%%%%%%%%%
\frame
{
  \frametitle{Linear Algebra}
  \begin{center}
    \includegraphics[width=\textwidth,trim=7.56in 0 0 0,clip]{libmesh_docs/classlibMesh_1_1SparseMatrix__inherit__graph}
  \end{center}
}



%%%%%%%%%%%%%%%%%%%%%%%%%%%%%%%%%%%%%%%%%%%%%%%%%
\frame
{
  \frametitle{I/O formats}
  \begin{center}
    \includegraphics[height=0.9\textheight]{libmesh_docs/mesh_io}
  \end{center}
}


%%%%%%%%%%%%%%%%%%%%%%%%%%%%%%%%%%%%%%%%%%%%%%%%%
\frame
{
  \frametitle{Domain Partitioning}
  \begin{center}
    \includegraphics[width=.3\textwidth]{part_trans}
    %\\
    \includegraphics[width=.3\textwidth]{streamtraces}
  \end{center}  

  \includegraphics[width=.65\textwidth]{libmesh_docs/partitioner}
}


%%%%%%%%%%%%%%%%%%%%%%%%%%%%%%%%%%%%%%%%%%%%%%%%%
\frame
{
  \frametitle{Discretization: Finite Elements}
  \begin{center}
    \includegraphics[width=0.9\textwidth,trim=7.4in 0 0 0,clip]{libmesh_docs/classlibMesh_1_1FEAbstract__inherit__graph}
  \end{center}
}      



%%%%%%%%%%%%%%%%%%%%%%%%%%%%%%%%%%%%%%%%%%%%%%%%%
\frame
{
  \frametitle{Algorithms: Error Estimation}
  \begin{center}
    \includegraphics[width=\textwidth]{libmesh_docs/error_estimation}
  \end{center}
}



%%%%%%%%%%%%%%%%%%%%%%%%%%%%%%%%%%%%%%%%%%%%%%%%%
\begin{frame}
\frametitle{Mesh Data Structures}
\begin{columns}
\column{.6\textwidth}
\begin{center}
\includegraphics[width=.95\textwidth]{MeshUML}
\end{center}
\column{.4\textwidth}
%\begin{block}{}
\begin{itemize}
\item \texttt{MeshBase} gives predicated iterators, node or element, all or active, global or local
\item \texttt{ReplicatedMesh} or \texttt{DistributedMesh} manages synchronized or distributed data
\end{itemize}

\includegraphics[width=.75\textwidth]{ParallelMesh3}
%\end{block}
\end{columns}

\end{frame}





% LocalWords:  nasablue
