

%=================================================================
% Outline
%=================================================================
%\section{Introduction}
%\input{outline_currentsection}
\input{outline}



\section{Introduction}
%%%%%%%%%%%%%%%%%%%%%%%%%%%%%%%%%%%%%%%%%%%%%%%%%
\commentout{
\frame
{
  \frametitle{Background}

  \begin{itemize}
  \item Modern simulation software is \emphcolor{complex}:
    \begin{itemize}
    \item Implicit numerical methods
    \item Massively parallel computers
    \item Adaptive methods
    \item Multiple, coupled physical processes
    \end{itemize}
    %\pause
  \item There are a host of existing software libraries that excel at treating various aspects of this complexity.
  \item Leveraging existing software whenever possible is the most efficient way to manage this complexity.

  \end{itemize}
}




%%%%%%%%%%%%%%%%%%%%%%%%%%%%%%%%%%%%%%%%%%%%%%%%%
\frame
{
  \frametitle{Background}

  \begin{itemize}
  \item Modern simulation software is \emphcolor{multidisciplinary}:
    \begin{itemize}
    \item Physical Sciences
    \item Engineering
    \item Computer Science
    \item Applied Mathematics
    \item \ldots
    \end{itemize}
  \item It is not reasonable to expect a single person to have all the necessary skills for developing \& implementing high-performance numerical algorithms on modern computing architectures.
  \item Teaming is a prerequisite for success.
  \end{itemize}
}
} % commentout


 

%%%%%%%%%%%%%%%%%%%%%%%%%%%%%%%%%%%%%%%%%%%%%%%%%
\frame
{
  \frametitle{Background}                 
  \begin{itemize}
    \item A large class of problems are amenable to \emphcolor{mesh based} simulation techniques.
      %% \begin{columns}[t]
      %%   \column{.5\textwidth}        
      %%   \fbox{\includegraphics[viewport=140 420 400 685,clip=true,height=1in]{domain2/domain2_input}}
      %%   \column{.5\textwidth}
      %%   \fbox{\includegraphics[height=1in,angle=-90]{discretized_domain}}
      %% \end{columns}
    \item Consider some of the major components such a simulation:
      \pause
      \begin{enumerate}
        \item Read the mesh from file
        \item Initialize data structures
        \item Construct a discrete representation of the governing equations
        \item Solve the discrete system
        \item Write out results
        \item Optionally estimate error, refine the mesh, and repeat
      \end{enumerate}

    \pause
    \item With the exception of step 3, the rest is \emph{independent} of the class of problems being solved.
    \pause
    \item This allows the major components of such a simulation to be abstracted \& implemented in a reusable software library.
  \end{itemize}
}


 

\subsection{The \libmesh{} Software Library}
%%%%%%%%%%%%%%%%%%%%%%%%%%%%%%%%%%%%%%%%%%%%%%%%%
\frame
{
  \frametitle{The \libmesh{} Software Library}
  \begin{itemize}
    \item In 2002, the \libmesh{} library began with these ideas in mind.
    \item Primary goal is to provide data structures and algorithms that can be shared by disparate physical applications, that may need some combination of
      \begin{itemize}
      \item Implicit numerical methods
      \item Adaptive mesh refinement techniques
      \item Parallel computing
      \end{itemize}
    \item Unifying theme: \emphcolor{mesh-based simulation of partial differential equations (PDEs)}.
  \end{itemize}
}



 

\subsection{Software Reusability}
%%%%%%%%%%%%%%%%%%%%%%%%%%%%%%%%%%%%%%%%%%%%%%%%%
\frame
{
  \frametitle{The \libmesh{} Software Library}

\begin{columns}
\column{.5\textwidth}
  \begin{block}{\libMesh{} Goals}
    \begin{itemize}
      \item Use by students, researchers, scientists, and engineers as a tool for \emphcolor{developing simulation codes} or as a tool for \emphcolor{rapidly implementing a numerical method}.
      \item \libMesh{} is not an application code.
      \item It does not ``solve problem XYZ.''
        \begin{itemize}
          \item It helps users develop applications to solve XYZ,
              quickly, with advanced algorithms on HPC platforms.
        \end{itemize}
      %\item It was initially targeted for finite element based simulations, but has been used for finite volume discretizations as well.
    \end{itemize}    
  \end{block}
\column{.45\textwidth}
  \begin{center}
    \includegraphics[width=.9\textwidth]{mytreeandroots_allnames}
  \end{center}
\end{columns}
} 


\begin{frame}{libMesh Community}
\begin{columns}
\column{.4\textwidth}
\begin{block}{Scope}
\begin{itemize}
\item Free, Open source
\begin{itemize}
\item LGPL2 for core
\end{itemize}
\item 70 Ph.D.\ theses, 721 papers (134 in 2017)
\item $\sim10$ current developers
\item $110-240$ current users?
\end{itemize}
\end{block}

\column{.6\textwidth}
\includegraphics[width=.45\textwidth]{ablating_hs_wbg}
\includegraphics[width=.25\textwidth]{sov}
\includegraphics[width=.3\textwidth]{marmot1b}
\end{columns}

\begin{columns}
\column{.35\textwidth}
\includegraphics[width=\textwidth]{libmesh_citations}

\column{.65\textwidth}
\begin{block}{Challenges}
\begin{itemize}
\item Radically different application types
\item Widely dispersed core developers
\begin{itemize}
\item INL, UT-Austin, U.Buffalo, JSC, MIT, Harvard, Argonne
\end{itemize}
\item OSS, commercial, private applications
\end{itemize}
\end{block}
\end{columns}

\end{frame}




