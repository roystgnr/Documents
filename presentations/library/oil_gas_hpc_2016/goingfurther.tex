\section{Going Further}

\subsection{Mesh Generation \& Importing}
\frame
{
  \begin{block}{Mesh Generation}
    \begin{itemize}
      \item You will need access to a complete mesh generation package to create complex meshes for use inside \libMesh{}.
      \item General process involves importing geometry, creating a volume mesh, assigning boundary conditions, and exporting the mesh.
      \item Recommended mesh generation packages:
        \begin{itemize}
          \item \texttt{gridgen/pointwise}: export mesh in \texttt{ExodusII} format.
          \item \texttt{gmsh}: open-source mesh generator. \libMesh{} supports \texttt{gmsh} format.
          \item \texttt{Cubit}: unstructured mesh generator from Sandia National Labs.
          \item Need another mesh format? \url{libmesh-users@lists.sourceforge.net}
        \end{itemize}
    \end{itemize}
  \end{block}
}


\subsection{Discontinuous Galerkin FEM}
\frame
{
  \begin{block}{Discontinuous Galerkin Support}
    \begin{itemize}
      \item \libMesh{} supports a rich set of discontinous finite element bases.
      \item For an example using interior penalty DG, see \texttt{\$LIBMESH\_DIR/examples/miscellaneous/ex5}
    \end{itemize}
  \end{block}
}

\subsection{Multiphysics Applications}
\frame
{
  \begin{block}{Multiphysics Support}
    \begin{itemize}
      \item Tightly coupled multiphysics:
        \begin{itemize}
          \item All variables should be placed in the same \texttt{System}.
          \item You can restrict variables to subdomains: c.f.\ \texttt{\$LIBMESH\_DIR/examples/subdomains/ex1}
        \end{itemize}
        \item Loosely coupled multiphysics:
          \begin{itemize}
            \item On the same \texttt{Mesh}, use different \texttt{Systems}
            \item On disjoint \texttt{Mesh}es, pass data via \texttt{MeshfreeInterpolation} objects.
          \end{itemize}
    \end{itemize}
  \end{block}
}



\subsection{FEMSystem Framework}
\frame
{
  \begin{block}{\texttt{FEMSystem}}
    \begin{itemize}
    \item \texttt{FEMSystem} provides an alternative programming interface specifically for finite element applications.
    \item Allows more direct interaction between solution algorithms and finite element approximation.
    \item See \texttt{\$LIBMESH\_DIR/examples/fem\_system}      
    \end{itemize}
  \end{block}
}


\subsection{Reduced Basis Modeling}
\frame
{
  \begin{block}{Reduced Basis Modeling}
    \begin{itemize}
      \item Recent effort led by David Knezevic to add certified reduced basis support to \libMesh{}.
      \item This functionality allows fine-scale solutions to be approximated in an optimal basis with an associated error estimate.
      \item See \texttt{\$LIBMESH\_DIR/examples/reduced\_basis}
    \end{itemize}
  \end{block}
}
